\documentclass{sig-alternate}
%\documentclass{acm_proc_article-sp}

\usepackage{etex}
\usepackage[english]{babel}
\usepackage[T1]{fontenc}
\usepackage[utf8]{inputenc}
\usepackage{microtype}
\usepackage{url}
\usepackage{doi}
\usepackage{hyperref}
\usepackage[]{color}

\definecolor{grey}{rgb}{0.5,0.5,0.5}
\definecolor{darkgreen}{cmyk}{0.7, 0, 1, 0.5}
\definecolor{LinkColor}{rgb}{0,0,0.5}

\hypersetup{%%Hyperref stuff
	pageanchor=false,
	pdfpagelabels=false,
	final,
    colorlinks=true, 
    linkcolor=LinkColor,
    citecolor=LinkColor,
    filecolor=LinkColor,
    menucolor=LinkColor,
    urlcolor=LinkColor,
	pdftitle={Crowdsourcing}
}

\begin{document}

\title{The scientific state of the art in Crowdsourcing}
\subtitle{An overview of Crowdsourcing, its different facets, challenges and criticisms}

\numberofauthors{6}

\author{%% Authors start here
\alignauthor
Diego Barbera\\ 
\affaddr{1327823}
%
\alignauthor
Viktor Kvaternjak\\
\affaddr{1328111}
%
\alignauthor
Andreas Ntaflos\\
\affaddr{0326302}
%
\and
%
\alignauthor
Richard Plangger\\ 
\affaddr{1025637}
%
\alignauthor
Christian Schnitzer\\ 
\affaddr{0828380}
%
\alignauthor
Konrad Steiner\\ 
\affaddr{0927159}
\and
%% Authors end here
}

\maketitle

\begin{abstract}
\end{abstract}

\section{Introduction}
% What? Term coined by Howe in Wired article howe2006rise
% Various, conflicting definitions: Wikipedia? Youtube? vs InnoCentive, iStockphoto, Amazon Mechanical Turk
% Proper definition by Estell{\'e}s-Arolas and Gonz{\'a}lez-Ladr{\'o}n-de-Guevara
% Purpose of paper:
%  - define what ``crowdsourcing'' really is, based on definition and characteristics by Estelles at al
%  - exemplary overview of crowdsourcing platforms and systems based on definition
%   - what is a CS platform, what isn't?
%  - what it takes to create a successful CS initiative
%  - examination of different types of crowd, incentives, etc and how they apply to different CS initiatives
%  - criticisms of CS

\section{Crowdsourcing defined}
% mostly based on estelles2012towards and brabham2008crowdsourcing
% define what ``crowdsourcing'' really is, based on definition and
%   characteristics by Estelles at al
% concept of ``wisdom of crowds'' and why a crowd may be smarter than its
%   smartest member; average vs aggregation
% distinction from Open Source
% examples of CS vs examples of Not-CS

\section{Overview of Crowdsourcing systems and platforms}
% mostly based on estelles2012towards
% exemplary overview of crowdsourcing platforms and systems based on definition
% what is a CS platform, what isn't?
% CS initiatives run on CS platforms
%   not all initiatives run on all platforms
% hint at topic of different types of crowds for different types of problems
% and tasks

\section{Different crowds for different problems}
% who makes up the crowd? what are their motivations?
% creative or intellectual problems vs repetitive and tedious problems
% community vs anonymity and micro-tasks
% amazon mturk vs innocentive/threadless
% importance or unimportance of heterogeneity of crowd
% hint at topic of quality control 

\section{Challenges in Crowdsourcing initiatives}
% kittur2008crowdsourcing, hossfeld2013crowdsourcing
% what it takes to create a successful CS initiative
% Designing tasks, spreading tasks to many workers, testing workers, avoiding
% abuse/scams, ensuring quality work

\section{Critical considerations of Crowdsourcing}
% manipulation, pranks, ``not invented here'', labour exploitation
% the ``crowd'' is not as diverse and heterogeneous as it might seem: white,
%   middle/upper class, English speaking, young, higher education, high-speed
%   internet connection
% the real winners in crowdsourcing are the platform providers, profiting from every problem and every solution
%

\section{Conclusion}

\nocite{*}
\bibliographystyle{abbrv}
\bibliography{crowdsourcing}


\end{document}

\documentclass{sig-alternate}
%\documentclass{acm_proc_article-sp}

\usepackage{etex}
\usepackage[english]{babel}
\usepackage[T1]{fontenc}
\usepackage[utf8]{inputenc}
\usepackage{microtype}
\usepackage{url}
\usepackage{doi}
\usepackage{hyperref}
\usepackage[]{color}
\usepackage[shortlabels]{enumitem}

\definecolor{grey}{rgb}{0.5,0.5,0.5}
\definecolor{darkgreen}{cmyk}{0.7, 0, 1, 0.5}
\definecolor{LinkColor}{rgb}{0,0,0.5}

\setlist{noitemsep}

\hypersetup{%%Hyperref stuff
	pageanchor=false,
	pdfpagelabels=false,
	final,
	colorlinks=true, 
	linkcolor=LinkColor,
	citecolor=LinkColor,
	filecolor=LinkColor,
	menucolor=LinkColor,
	urlcolor=LinkColor,
	pdftitle={Crowdsourcing}
}

\begin{document}

\title{The scientific state of the art in Crowdsourcing}
\subtitle{An overview of Crowdsourcing, its different facets, challenges and criticisms}

\numberofauthors{6}

\author{%% Authors start here
	\alignauthor
	Diego Barbera\\ 
	\affaddr{1327823}
%
	\alignauthor
	Viktor Kvaternjak\\
	\affaddr{1328111}
%
	\alignauthor
	Andreas Ntaflos\\
	\affaddr{0326302}
%
	\and
%
	\alignauthor
	Richard Plangger\\ 
	\affaddr{1025637}
%
	\alignauthor
	Christian Schnitzer\\ 
	\affaddr{0828380}
%
	\alignauthor
	Konrad Steiner\\ 
	\affaddr{0927159}
	\and
%% Authors end here
}

\maketitle

\begin{abstract}
	Lorem ipsum dolor sit amet, consectetur adipiscing elit. Etiam congue
	lorem ut lorem lacinia, at pharetra dui tincidunt. Curabitur blandit enim
	nulla, vitae varius risus aliquam ut. Etiam at leo felis. Suspendisse
	potenti. Sed malesuada lacus vestibulum cursus ornare. Sed eu purus
	hendrerit, consequat ante sagittis, auctor quam. Donec elit massa,
	pellentesque ac gravida vitae, pellentesque a nulla. In molestie dolor
	tristique luctus adipiscing. Nunc iaculis aliquet leo vel vestibulum.
	Integer eu lacus vehicula, vestibulum mi at, tincidunt leo. Sed ornare
	nunc vel lectus sagittis, auctor ultricies massa porttitor. Nullam sit
	amet condimentum lacus, sed vehicula enim. Aliquam in sem elit. Duis
	pretium mi at dignissim placerat.
\end{abstract}

\section{Introduction}

The term \emph{crowdsourcing} has been coined by Jeff Howe in
2006~\cite{howe2006rise} as a portmanteau of \emph{crowd} and
\emph{outsourcing}. He describes it generally as outsourcing work to an
``undefined, generally large group of people in the form of an open
call''~\cite{howe2009crowdsourcing}. The term and the underlying practice are
still quite young and the theoretical backgrounds---What is a crowd? What
motivates the crowd? Why do some crowdsourcing systems and initiatives thrive
while others fail miserably?---are still being researched, for example in the
works of Daren Brabham~\cite{brabham2008crowdsourcing, brabham2008moving,
brabham2010moving}. This is why in scientific literature we find various,
often directly conflicting definitions of
crowdsourcing~\cite{estelles2012towards}, as well as conflicting
classifications of existing systems and initiatives; some authors see
Wikipedia or Youtube as crowdsourcing examples while others claim the exact
opposite~\cite{estelles2012towards}.

In this paper we give an overview of crowdsourcing by discussing an integrated
definition of the term and practice as put forth by Estell{\'e}s-Arolas and
Gonz{\'a}lez-Ladr{\'o}n-de-Guevara in~\cite{estelles2012towards} and examining
current crowdsourcing systems and platforms based on that definition. We also
compare the motivations and composition of different kinds of crowds for
different kinds of problems and how these crowds often differ widely in these
regards. Next we describe broadly how to design a crowdsourcing initiative to be
successful and which steps can be taken to control reliability and quality of
work. Finally we discuss some critical aspects of and problems with
crowdsourcing which are often overlooked or ignored in current literature. 

% What? Term coined by Howe in Wired article howe2006rise
% Various, conflicting definitions: Wikipedia? Youtube? vs InnoCentive, iStockphoto, Amazon Mechanical Turk
% Proper definition by Estell{\'e}s-Arolas and Gonz{\'a}lez-Ladr{\'o}n-de-Guevara
% Purpose of paper:
%  - define what ``crowdsourcing'' really is, based on definition and characteristics by Estelles at al
%  - exemplary overview of crowdsourcing platforms and systems based on definition
%   - what is a CS platform, what isn't?
%  - what it takes to create a successful CS initiative
%  - examination of different types of crowd, incentives, etc and how they apply to different CS initiatives
%  - criticisms of CS

\section{Crowdsourcing defined}

The most comprehensive definition of crowdsourcing has been compiled by
Estell{\'e}s-Arolas and Gonz{\'a}lez-Ladr{\'o}n-de-Guevara. In their
work~\cite{estelles2012towards} the authors analysed a large number of
existing definitions and extracted various characteristics that apply to a
crowdsourcing system, resulting in the following: 
\begin{quotation}
	Crowdsourcing is a type of participative online activity in which an
	individual, an institution, a non-profit organization, or company proposes to
	a group of individuals of varying knowledge, heterogeneity, and number, via a
	flexible open call, the voluntary undertaking of a task. The undertaking of
	the task, of variable complexity and modularity, and in which the crowd should
	participate bringing their work, money, knowledge and/or experience, always
	entails mutual benefit. The user will receive the satisfaction of a given type
	of need, be it economic, social recognition, self-esteem, or the development
	of individual skills, while the crowdsourcer will obtain and utilize to their
	advantage that what the user has brought to the venture, whose form will
	depend on the type of activity undertaken.
\end{quotation}

Based on that definition all of the following characteristics, as identified
in~\cite{estelles2012towards} apply to a crowdsourcing system:

\begin{enumerate}[(a)]
	\item There is a clearly defined crowd\label{a}
	\item There exists a task with a clear goal\label{b}
	\item The recompense received by the crowd is clear\label{c}
	\item The crowdsourcer is clearly identified\label{d}
	\item The compensation received by the crowdsourcer is clearly defined\label{e}
	\item It is an online assigned process of participative type\label{f}
	\item It uses an open call of variable extent\label{g}
	\item It uses the internet\label{h}
\end{enumerate}

Having identified the characteristics that describe a crowdsourcing system it
is now easy to see that neither Wikipedia, nor Youtube qualify as such. For
Wikipedia the missing characteristics are~\ref{d},~\ref{e} and~\ref{g},
while for Youtube only~\ref{a} and~\ref{h} even apply; all others do not.

% mostly based on estelles2012towards and brabham2008crowdsourcing
% define what ``crowdsourcing'' really is, based on definition and
%   characteristics by Estelles at al
% concept of ``wisdom of crowds'' and why a crowd may be smarter than its
%   smartest member; average vs aggregation
% distinction from Open Source
% examples of CS vs examples of Not-CS

\section{Overview of Crowdsourcing systems and platforms}
% mostly based on estelles2012towards
% exemplary overview of crowdsourcing platforms and systems based on definition
% what is a CS platform, what isn't?
% CS initiatives run on CS platforms
%   not all initiatives run on all platforms
% hint at topic of different types of crowds for different types of problems
% and tasks

\section{Different crowds for different problems}
% who makes up the crowd? what are their motivations?
% creative or intellectual problems vs repetitive and tedious problems
% community vs anonymity and micro-tasks
% amazon mturk vs innocentive/threadless
% importance or unimportance of heterogeneity of crowd
% hint at topic of quality control 

\section{Challenges in Crowdsourcing initiatives}
% kittur2008crowdsourcing, hossfeld2013crowdsourcing
% what it takes to create a successful CS initiative
% Designing tasks, spreading tasks to many workers, testing workers, avoiding
% abuse/scams, ensuring quality work

\section{Critical considerations of Crowdsourcing}
% manipulation, pranks, ``not invented here'', labour exploitation
% the ``crowd'' is not as diverse and heterogeneous as it might seem: white,
%   middle/upper class, English speaking, young, higher education, high-speed
%   internet connection
% the real winners in crowdsourcing are the platform providers, profiting from every problem and every solution
%

\section{Conclusion}

\nocite{*}
\bibliographystyle{abbrv}
\bibliography{crowdsourcing}


\end{document}
